% Project - Technical Documentation
%
% Data Acquisition Technologies and Sensor Networks
% Jacobs University Bremen
% Supervisor: Prof. Dr. Fangning Hu
%
% Created on November 20, 2019
%
% Authors:
%   Ralph Florent <r.florent@jacobs-university.de>
%   Diogo Cosin <d.ayresdeoliveira@jacobs-university.de>
%   Eno Ciraku <e.ciraku@jacobs-university.de>

% ==============================================================================
% START: Methodology
% ==============================================================================

\section{Methodology}
In this section, we describe the techniques and strategies as well as the procedural methods used to implement the core functionality of this project. This description includes the components of the system, the workflow scheme, the third-party libraries, the algorithm and data structure, and finally, the code implementation.

It is important to stand out the fact that we use external resources to come up with daily results in terms of programming tools and full-on working devices. That is, before adventuring into using the materials and devices as well as certain software tools, different research sources were consulted. Some of these resources are: the datasheets of the prebuilt modules of the materials, the assistance of the instructor and teaching assistants (TAs), search engines (mainly Google Search/Internet), and finally some related books and references (e.g. materials provided by the professor).

Being exceptionally helpful, these resources have been the source of truth for any decision-making regarding the correct use of the hardware modules.

\subsection{Components of the prototype}
Roughly speaking, Smart Outlet comprises 3 (three) principal modules:
\begin{itemize}
    \item \textbf{\textit{Webduino}}: a low-level, modular circuitry formed by an Arduino and a network of sensors. The Arduino board, a microcontroller, acts as a supervisor of micro tasks. Being the core component of the hardware systems, it controls the different input/ouput functions of the connected chips.
    \item \textbf{\textit{Web API}}: an API service attending HTTP requests from the \emph{webduino} (its only consumer). It coordinates the communication between the Wi-Fi module of the webduino (client) over the air medium (wireless) and the available API resources on an HTTP server. The API service contains various layers of interactons, including the database for data persistence.
    \item \textbf{\textit{Web APP}}: (short for web application) a single page application (SPA) to visualize the historical content or performed actions during the webduino's operations. It allows user-friendly interactions between an end-user and the prototype itself. The web app is also responsive. That is, it can be accessed and used via mobile devices (tablets, smartphones, etc.).
\end{itemize}

Each one of these modules deep down contains a set of characteristics that requires more than a brief description to highlight their corresponding functionality. However, in this document, we intend to only explain how to connect them together and make them work properly as a whole. We indeed provide full access to the online repository as specified in Appendix \ref{sec:code-repo} so that anyone can dig any further into the datasheets if needs be.

\subsection{Workflow scheme}

\subsection{Algorithm and data structure}

\subsection{Code implementation}
% ==============================================================================
% END: Methodology
% ==============================================================================