% Project - Technical Documentation
%
% Data Acquisition Technologies and Sensor Networks
% Jacobs University Bremen
% Supervisor: Prof. Dr. Fangning Hu
%
% Created on November 13, 2019
%
% Authors:
%   Ralph Florent <r.florent@jacobs-university.de>
%   Diogo Cosin <d.ayresdeoliveira@jacobs-university.de>
%   Eno Ciraku <e.ciraku@jacobs-university.de>
%
% Introductory part of the documentation

% ==============================================================================
% START: Intro
% ==============================================================================

\section{Introduction}
\label{sec:intro}

In this report, the solution, specification, and outcomes of the project of a
Smart Outlet are presented as part of the course \textit{Data Acquisition
Technologies and Sensor Networks} grading criteria. The course proposes to
design a sensor network communicating wirelessly via a web interface with a
database. The recorded data should be visualized graphically through a
web interface.

In this section, we describe our proposed solution, its motivation, and comment
on this report organization.

\subsection{Solution Proposal}

The so-called Smart Outlet (SO) is a power outlet remotely controlled via a webpage
interface. The solution offers a switch \textit{on/off}, which can be toggled
through the previously mentioned webpage. The same webpage also provides a
monitoring feature that allows the users to verify the states of the SO
throughout time tabular and graphically.

The components of the macro solution of the project are given follows:

\begin{itemize}
    \item \textbf{Hardware component:} the power outlet, the controller device,
    and the wireless communication interface;
    \item \textbf{Webpage:} web-interface for controlling the SO states;
    \item \textbf{Database:} relational database for storing the states of the SO.
\end{itemize}

Further details of the solution are introduced in Sections
\ref{sec:inst-resour} and \ref{sec:methodology}.

In this project, we decided to constrain the SO range to a Wireless Local Area
Network (WLAN). However, this limitation can be overcome by exposing the SO
solution in a public HTTP interface. This feature may be implemented in a
possible next step of this project (out of scope in this report).

\subsection{Motivation}

Besides being a requirement to the successful completion of the course
\textit{Data Acquisition Technologies and Sensor Networks}, building the SO
offers many advantages to a possible end user. For these reasons, we
decided to implement a prototype for this solution. These advantages are
described as follows:

\begin{itemize}
    \item \textbf{Convenient usability:} the SO can be controlled from any place
    considering that the webpage may be exposed publicly via HTTP requests;
    \item \textbf{Energy saving:} the user can track the SO state and turn it
    off when considering that it has been unnecessarily used;
    \item \textbf{Multiple users:} the solution enables the access of different
    users through different platforms by only requiring a device that can access
    and visualize webpages exposed on the internet. In other words, users with
    computers, smartphones, and tablets can control the SO.
\end{itemize}

\subsection{Note on the Report Structure}

Before the next Sections of the report, in which technical details and results are
presented, let us briefly enlighten how this report is structured. The structure
follows: 

\begin{itemize}
    \item \textbf{Theoretical review:} theoretical and technical details of the
    chosen hardware components are briefly presented;
    \item \textbf{Instrumentation \& resources:} the integration of the hardware
    and software of the SO prototype is detailed;
    \item \textbf{Methodology:} the techniques, strategies, and procedural
    methods used to implement the core functionality of the project are
    presented;
    \item \textbf{Results \& discussions:} results obtained with the SO
    prototype implementation.
\end{itemize}

%  Do not forget to mention here: goals, problem-solution aspects, keywords
%  (glossary)

% ==============================================================================
% END: Intro
% ==============================================================================