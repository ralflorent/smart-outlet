% Project - Technical Documentation
%
% Data Acquisition Technologies and Sensor Networks
% Jacobs University Bremen
% Supervisor: Prof. Dr. Fangning Hu
%
% Created on November 13, 2019
%
% Authors:
%   Ralph Florent <r.florent@jacobs-university.de>
%   Diogo Cosin <d.ayresdeoliveira@jacobs-university.de>
%   Eno Ciraku <e.ciraku@jacobs-university.de>
%
% Appendices part of the documentation

% ==============================================================================
% START: Appendix
% ==============================================================================

\clearpage
\appendix
\begin{appendices}
    \section{Code Repository}
    \label{sec:code-repo}

    All the code implemented during the execution of the prototype described in this report is available on the GitHub repository \href{https://github.com/ralflorent/smart-outlet}{https://github.com/ralflorent/smart-outlet}.

    \section{Understand the Repository}
    It is highly recommended to check and read the markdown files (e.g. \emph{README.md}) to encourage further understanding of every part of the repository. It is a self-sufficient, self-explanatory repository where every taken step is detailed with sustainable reasons.

    We also try to follow the best practices by using the recommendations of the open-source community. For example, observe how we use the commit messages, the file structures, the naming conventions, and so forth.

    Finally, we use up-to-date tools and software so that we can take advantage of their fully-available features. To illustrate our point, we use the last version Angular Framework to develop the web application. That is, supporting and upgrading this framework are indisputably and relatively easy.

\end{appendices}
% ==============================================================================
% END: Scripts
% ==============================================================================