% Project - Technical Documentation
%
% Data Acquisition Technologies and Sensor Networks
% Jacobs University Bremen
% Supervisor: Prof. Dr. Fangning Hu
%
% Created on November 13, 2019
%
% Authors:
%   Ralph Florent <r.florent@jacobs-university.de>
%   Diogo Cosin <d.ayresdeoliveira@jacobs-university.de>
%   Eno Ciraku <e.ciraku@jacobs-university.de>
%
% 1) Introduction
% 2) Theoretical background
% 3) Instrumentation (software and tools)
% 4) Methods (Procedure)
% 5) Results, Discussions
% 6) Conclusion

% ==============================================================================
% START: Main Content
% ==============================================================================

% Hardware and Software Resources
% Project - Technical Documentation
%
% Data Acquisition Technologies and Sensor Networks
% Jacobs University Bremen
% Supervisor: Prof. Dr. Fangning Hu
%
% Created on November 20, 2019
%
% Authors:
%   Ralph Florent <r.florent@jacobs-university.de>
%   Diogo Cosin <d.ayresdeoliveira@jacobs-university.de>
%   Eno Ciraku <e.ciraku@jacobs-university.de>
%
% Intrumentation and Resources of the documentation

% ==============================================================================
% START: Intrumentation & Resources
% ==============================================================================

\section{Intrumentation \& Resources}
Recalling that the SO prototype is the complete integration of a set of hardware and software, we detail in this section more in-depth specifications on them. Below are enlisted the tools used to set up and carry out successfully the current version of the project.

\subsection{Hardware and other materials}
The SO hardware refers to the physical components that add up to the \emph{Webduino} module. Although there are many possible options in choosing different kinds of hardware to build up the circuitry, we opt for the most reasonable\footnote{Building a separate module from scratch requires time and work, plus other tasks to refine a working component.} choice, which is to reuse already-prebuilt modules and integrate them into one. Hence, in Table \ref{table:hardware-and-materials} are listed the items with their corresponding details.
\begin{table}[!ht]
    \begin{center}
        \begin{tabular}{ |l|l|l|l| }
            \hline
            \multicolumn{4}{ |c| }{ \textbf{ Hardware \& Other Materials }} \\
            \hline % Table headers
             & \textbf{Series} & \textbf{Quantity} & \textbf{Cost}  \\ [0.5ex]
            \hline % Table body (row-wise contents)
            \textbf{\textit{Microcontroller Board}} & MEGA 2560 R3 & 1 & \euro{ 14,99 }  \\
            \hline
            \textbf{\textit{4-Channel Relay}} & GE-EL-SM-006 & 1 & \euro{ 6,99 }  \\
            \hline
            \textbf{\textit{Wi-Fi Module}} & ESP8266-01 & 1 & \euro{ 6,99 } \\
            \hline
        \end{tabular}
        \caption{Detailed information on the hardware and materials used for the SO prototype.}
        \label{table:hardware-and-materials}
    \end{center}
\end{table}

Additionally, other useful materials such as:
\begin{itemize}
    \item 12x male-to-male jumper cables (20 cm)
    \item 12x male-to-female jumper cables (20 cm)
    \item 12x female-to-female jumper cables (20 cm)
    \item 1x access point or router (tp-link TL-WR940N)
    \item 2x computing devices or laptop (Lenovo T460p and Macbook Air)
\end{itemize}
remain useful to connect the different components and have them work as one stable module. More technical specifications are given on each one of the materials and devices, including their working conditions, in the section \emph{State of Art}.

% TODO: touch the budget here

\subsection{Tools and software}
Next, we present the set of tools and software that are used at the time of implementing the prototype:
\begin{itemize}
    \item Operating systems (GNU/Linux, Mac OS, and Windows)
    \item Visual Studio Code (lightweight text editor)
    \item Git\footnote{Git is also available as a bash emulation for other platforms for free (e.g., Git Bash for Windows).} (version control)
    \item GitHub (web-based hosting service for Git versioning system)
    \item Jupyter Notebook (workspace for scripting and simulation)
    \item Arduino IDE (development environment for Arduino boards)
\end{itemize}

% TODO: list current versions in table

Regarding the software versions, it is highly recommended to use the exact versions mentioned in Table 2 to avoid conflicts and compiling errors. On the other hand, the developer can always dig into the breaking changes (if that is the case) that might requires to refactor part of the implementation to have a fully working prototype. However, it is recommended to check the changelog of the updates/releases, if any.

\noindent
Note that some tools mentioned above are just a matter of personal preferences. Other preferred options are more than welcome as long as the developer keeps in mind development speed and productivity. For example, many developers would choose \href{https://www.sublimetext.com/}{Sublime} over \emph{Visual Studio Code}. But they both end up facilitating the same routine: text edition.

\subsection{Programming languages}
Finally, we use the following programming languages:
\begin{itemize}
    \item C/C++ (for the webduino)
    \item Python (for the web API service)
    \item Angular Framework - JavaScript/TypeScript (for the web application)
\end{itemize}

\noindent
\textbf{Important}: \textit{Though we highly recommend that the exact versions of the software and the exact series/models of the materials/devices are used to test out or replicate this project, keep in mind that these hardware might no longer be available in the market as well as the software components might be outdated at some point in time. If that is the case, stay alerted to the updates as we intend to support this project until 2022.}

% ==============================================================================
% END: Intrumentation & Resources
% ==============================================================================

% Methodology
% Project - Technical Documentation
%
% Data Acquisition Technologies and Sensor Networks
% Jacobs University Bremen
% Supervisor: Prof. Dr. Fangning Hu
%
% Created on November 20, 2019
%
% Authors:
%   Ralph Florent <r.florent@jacobs-university.de>
%   Diogo Cosin <d.ayresdeoliveira@jacobs-university.de>
%   Eno Ciraku <e.ciraku@jacobs-university.de>

% ==============================================================================
% START: Methodology
% ==============================================================================

\section{Methodology}
In this section, we describe the techniques and strategies as well as the procedural methods used to implement the core functionality of this project. This description includes the components of the system, the workflow scheme, the third-party libraries, the algorithm and data structure, and finally, the code implementation.

It is important to stand out the fact that we use external resources to come up with daily results in terms of programming tools and full-on working devices. That is, before adventuring into using the materials and devices as well as certain software tools, different research sources were consulted. Some of these resources are: the datasheets of the prebuilt modules of the materials, the assistance of the instructor and teaching assistants (TAs), search engines (mainly Google Search/Internet), and finally some related books and references (e.g. materials provided by the professor).

Being exceptionally helpful, these resources have been the source of truth for any decision-making regarding the correct use of the hardware modules.

\subsection{Components of the prototype}
Roughly speaking, Smart Outlet comprises 3 (three) principal modules:
\begin{itemize}
    \item \textbf{\textit{Webduino}}: a low-level, modular circuitry formed by an Arduino and a network of sensors. The Arduino board, a microcontroller, acts as a supervisor of micro tasks. Being the core component of the hardware systems, it controls the different input/ouput functions of the connected chips.
    \item \textbf{\textit{Web API}}: an API service attending HTTP requests from the \emph{webduino} (its only consumer). It coordinates the communication between the Wi-Fi module of the webduino (client) over the air medium (wireless) and the available API resources on an HTTP server. The API service contains various layers of interactons, including the database for data persistence.
    \item \textbf{\textit{Web APP}}: (short for web application) a single page application (SPA) to visualize the historical content or performed actions during the webduino's operations. It allows user-friendly interactions between an end-user and the prototype itself. The web app is also responsive. That is, it can be accessed and used via mobile devices (tablets, smartphones, etc.).
\end{itemize}

Each one of these modules deep down contains a set of characteristics that requires more than a brief description to highlight their corresponding functionality. However, in this document, we intend to only explain how to connect them together and make them work properly as a whole. We indeed provide full access to the online repository as specified in Appendix \ref{sec:code-repo} so that anyone can dig any further into the datasheets if needs be.

\subsection{Workflow scheme}

\subsection{Algorithm and data structure}

\subsection{Code implementation}
% ==============================================================================
% END: Methodology
% ==============================================================================

% Results and Discussions
\section{Results and Discussions}
\label{sec:res-and-disc}

In this report, the different, relevant aspects of the project prototype called
Smart Outlet have been presented and described. The central component of the
project is the Arduino, which acts as micro-task supervisor and controls the
input/output functions of the chips as described in Section
\ref{sec:methodology}. Figure \ref{fig:proto} shows the prototype
assembled in the laboratory environment.

\begin{figure}[ht!]
    \centering
    \includegraphics[scale=0.15]{proto/sample-2.jpg}
    \caption{Assembled prototype.}
    \label{fig:proto}
\end{figure}

Other important aspects of this work include the architecture and design of
the different components used to make the prototype work. Knowledge of
Computer Networking, Web API, Web APP and Sensor Circuitry were fundamental in
building a complex module with the intended functionality.

To complete the project, trial and error approaches were employed in
gaining a greater control over each of our different components and then we
proceeded to merge the functionalities of the components. This part
also required trial and error as the interconnection of the modules not always
worked out the way they were expected to.

The prototype works as described in Section \ref{sec:methodology}. The user is able, 
either through a smartphone or a PC to manipulate the Smart Outlet by
switching it ON/OFF with the ultimate goal being a better energy consumption.
The user can also view the status of the outlet in real-time and further view a
history of the working of the outlet as shown in Figure \ref{fig:ui-history}.
% ==============================================================================
% END: Main Content
% ==============================================================================