% Project - Technical Documentation
%
% Data Acquisition Technologies and Sensor Networks
% Jacobs University Bremen
% Supervisor: Prof. Dr. Fangning Hu
%
% Created on November 20, 2019
%
% Authors:
%   Ralph Florent <r.florent@jacobs-university.de>
%   Diogo Cosin <d.ayresdeoliveira@jacobs-university.de>
%   Eno Ciraku <e.ciraku@jacobs-university.de>
%
% Instrumentation and Resources of the documentation

% ==============================================================================
% START: Intrumentation & Resources
% ==============================================================================

\section{Intrumentation \& Resources}
\label{sec:inst-resour}
Recalling that the SO prototype is the complete integration of a set of hardware and software, we detail in this section more in-depth specifications on them. Below are enlisted the tools used to set up and carry out successfully the current version of the project.

\subsection{Hardware and other materials}
The SO hardware refers to the physical components that add up to the \emph{Webduino} module. Although there are many possible options in choosing different kinds of hardware to build up the circuitry, we opt for the most reasonable\footnote{Building a separate module from scratch requires time and work, plus other tasks to refine a working component.} choice, which is to reuse already-prebuilt modules and integrate them into one. Hence, in Table \ref{table:hardware-and-materials} are listed the items with their corresponding details.
\begin{table}[!ht]
    \begin{center}
        \begin{tabular}{ |l|l|l|l| }
            \hline
            \multicolumn{4}{ |c| }{ \textbf{ Hardware \& Other Materials }} \\
            \hline % Table headers
             & \textbf{Models} & \textbf{Quantity} & \textbf{Cost}  \\ [0.5ex]
            \hline % Table body (row-wise contents)
            \textbf{\textit{Microcontroller Board}} & MEGA 2560 R3 & 1 & \euro{ 14,99 }  \\
            \hline
            \textbf{\textit{4-Channel Relay}} & GE-EL-SM-006 & 1 & \euro{ 6,99 }  \\
            \hline
            \textbf{\textit{Wi-Fi Module}} & ESP8266-01 & 1 & \euro{ 6,99 } \\
            \hline
        \end{tabular}
        \caption{Detailed information on materials and hardware used for the SO prototype.}
        \label{table:hardware-and-materials}
    \end{center}
\end{table}

Additionally, other useful materials such as:
\begin{itemize}
    \item 12x male-to-male jumper cables (20 cm)
    \item 12x male-to-female jumper cables (20 cm)
    \item 12x female-to-female jumper cables (20 cm)
    \item 1x access point or router (tp-link TL-WR940N)
    \item 2x computing devices or laptop (Lenovo T460p and Macbook Air)
\end{itemize}
remain useful to connect the different components and have them work as one stable module. More technical specifications are given on each one of the materials and devices, including their working conditions, in Section \ref{sec:state-of-art}.

\noindent
\textbf{Notes}: Observe that the monetary budget reaches the sum of \euro{28,97} for only one SO prototype that eventually includes the monitoring and control of 4 (four) power outlets. Most of the materials were bought online on \href{https://www.amazon.de/}{Amazon.de} and some of them, provided by the \emph{Data Acquisition} lab.

Considering that the main purpose of building this prototype is educational, we
do not account for commercial goals. That leaves room for improvement in the
future if we want to aim for industrialization of the project. That signifies
that, at some point, we might need to rethink the choice of tools and software
to optimize the build of the hardware module, which may, in turn, favor
a cost reduction in our financial records.

\subsection{Tools and software}
As for the programming tools used to create, debug, maintain the code of the Smart
Outlet at both hardware- and software-level, we use free programs yet efficient
that are available for most of the OS\footnote{OS: Operating Systems like
Windows, Mac OS, Linux, and so on.} platforms. These programs are of different
types such as code editors (e.g Visual Studio Code), compilers (e.g LaTeX),
online platforms (e.g GitHub), and so forth. With that being said, we present
the set of tools and software used at the time of implementing the prototype:
\begin{itemize}
    \item Operating systems (GNU/Linux, Mac OS, and Windows)
    \item Visual Studio Code (lightweight text editor)
    \item Git\footnote{Git is also available as a bash emulation for other platforms for free (e.g., Git Bash for Windows).} (version control)
    \item GitHub (web-based hosting service for Git versioning system)
    \item Jupyter Notebook (workspace for scripting and simulation)
    \item Arduino IDE (development environment for Arduino boards)
\end{itemize}

\begin{table}[!ht]
    \begin{center}
        \begin{tabular}{ |l|l|l|l| }
            \hline
            \multicolumn{4}{ |c| }{ \textbf{Tools \& Software}} \\
            \hline % Table headers
             & \textbf{Versions} & \textbf{Sources} & \textbf{Cost}  \\ [0.5ex]
            \hline % Table body (row-wise contents)
            \textbf{\textit{Visual Studio Code}} & 1.40.1 & See \cite{vscode} & Free  \\
            \hline
            \textbf{\textit{Git}} & 2.7.4 & Built-in Linux program & Free  \\
            \hline
            \textbf{\textit{GitHub}} & N/A & See \cite{github} & 5 free users  \\
            \hline
            \textbf{\textit{Python}} & 3.7.1 & See \cite{python} & Free  \\
            \hline
            \textbf{\textit{Jupyter Notebook}} & 5.7.4 & See \cite{jupyternb} & Free  \\
            \hline
            \textbf{\textit{Arduino IDE}} & 1.8.10 & See \cite{arduinoide} & Free  \\
            \hline
        \end{tabular}
        \caption{Detailed information on the tools and software used for the Smart Outlet coding procedure and the technical documentation.}
        \label{table:tools-and-software}
    \end{center}
\end{table}

Regarding the software versions, it is highly recommended to use the exact versions mentioned in Table \ref{table:tools-and-software} to avoid conflicts and compiling errors. On the other hand, the developer may want to dig into the breaking changes (if that is the case) that most of the time require to refactor parts of the implementation to have a fully working prototype. However, it is also recommended to check the \emph{changelog} of the updates or releases, if there are any.

Note that some tools mentioned above are just a matter of personal preferences. Other preferred options are more than welcome as long as the developer keeps in mind the development speed and productivity. For example, many developers would choose \href{https://www.sublimetext.com/}{Sublime} over \emph{Visual Studio Code}. But they both end up facilitating the same routine: text edition.

\subsection{Programming languages}
Finally, we use the following programming languages:
\begin{itemize}
    \item C/C++ (for the webduino)
    \item Python with the micro web framework Flask (for the web API service)
    \item Angular Framework - JavaScript/TypeScript (for the web application)
\end{itemize}

\noindent
\textbf{Important}: \textit{Though we highly recommend that the exact versions of the software and the exact models of the materials and devices are used to test out or replicate this project, keep in mind that these hardware might no longer be available in the market as well as the software components might be outdated at some point in time. If that is the case, stay alert to the updates as we intend to support this project until 2022.}

% ==============================================================================
% END: Intrumentation & Resources
% ==============================================================================